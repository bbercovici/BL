\section{Variational Bayes Learning of GM models}
%\EditMC{
%\begin{itemize}
% \item \textbf{Report the final number of non-zero mixture components determined by VB for each location l, 
% and the number of VB iterations required until convergence in each case.}
% \item  \textbf{Be sure to describe how you initialized the VB iterations and selected the upper bound Ml for each location.}
%\end{itemize}
%}
In this section, the Variational Bayes (VB) iterative process is applied in order to infer %learn 
the Gaussian mixture models of L locations separately. 
The locations considered are four, having been labeled as: 
AVS laboratory, ORCCA laboratory, corridor office and corridor ORCCA.

For the data set ($\bm Y$) of each location we construct a Gaussian mixture model 
and assume that the parameters ($\bm\pi$, $\bm\mu$, $\bm\Lambda$), the cluster assignments ($\bm Z$) 
and the true number of clusters are unknown.
\begin{figure}[!h]
	\centerline{ \includegraphics[scale=0.6]{FiguresC/model_plate}}
	\caption{
	Bayes net for fully Bayesian GM modeling of location-dependent essential features $y_i^{l}$. 
	}
 	\label{fig:model_plate}
\end{figure}
The cluster assignments are categorical variables, and the cluster probabilities are given a Dirichlet prior distribution, as shown Fig.~\ref{fig:model_plate} .
The unknown cluster means and precision matrices are given Gaussian and Wishart prior distributions respectively.


%For each location, an upper bound on the number of mixture components is assumed. 
Each location model uses a maximum number M clusters 
but the effective number of clusters will be determined automatically:
the VB algorithm knocks out the mixture components of the model 
that do not contribute on explaining the data, by 
updates of the responsibilities and the hyper-parameter values, which are coupled.

The prior value of the hyper-parameters shown in the model plate of Fig.~\ref{fig:model_plate} 
has been set as follows:
\begin{subequations}
\begin{align}
	\bm{m}_0 &= \bm{0}_{D\times 1} \\
	B_0 &= 0.01 \\
	v_0 &= D \\
	W_0 &= 0.125 \bm{I}_{D\times D}
\end{align}
\end{subequations}
where $D=8$ is the dimensionality of $y_i^{l}$.
It is important to mention that before running the VB algorithm, 
the symmetry in the model is broken by a random initialization of the cluster assignments. 
Without the random initialization, the clusters would not be separated. 
The VB algorithm updates the variables in turns and is run for 200 iterations or until convergence.


%%%%%%%%%%%%%%%%%%%%%%%%%%%%%%%%%%%%%%%%%%%%%%%%%%%%%%
\subsection{Results for M = 10 clusters}
With an initial assumption of a  maximum of $M=10$ clusters in all the location models, 
Fig.~\ref{fig:Responsibilities_M10} shows the normalized responsibilities of each mixture component k. 
%In turn, Fig.~\ref{fig:LowerBound} shows the evolution of the lower bound evaluation until convergence.
From these results it is observed that: 
\begin{itemize}
	\item{AVS laboratory (M = 10): }
	All the components seem to somehow contribute to explaining the observed data set . 
	Converged at iteration 23.
	\item{ORCCA laboratory (M = 10): }
	$6$ components are reasonably relevant to explain the observed data. 
	Converged at iteration 28.
	\item{Corridor office (M = 10): }
	About $8\sim9$ components contribute to explaining the observed data set.
	Converged at iteration 26.
	\item{Corridor ORCCA (M = 10): }
	All the components seem to somehow contribute to explaining the observed data set. 
	Converged at iteration 38.
\end{itemize}
\begin{figure}[!h]
	\centering
	\subfigure[$N_k$ for location: AVS laboratory. M = 10.]
	{\label{fig:Nk_AVS}
	\includegraphics[]{FiguresC_M10/Nk_norm_AVS}} 
	\subfigure[$N_k$ for location: ORCCA laboratory. M = 10.]
	{\label{fig:Nk_ORCCA} 
	\includegraphics[]{FiguresC_M10/Nk_norm_ORCCA}} 
	\subfigure[$N_k$ for location: corridor office. M = 10.]
	{\label{fig:Nk_corr_office}
	\includegraphics[]{FiguresC_M10/Nk_norm_corr_office}} 
	\subfigure[$N_k$ for location: corridor ORCCA. M = 10.]
	{\label{fig:Nk_corr_ORCCA}
	\includegraphics[]{FiguresC_M10/Nk_norm_corr_ORCCA}} 
	%
	\caption{Normalized responsibilities $N_k$ of each mixand $k$ in the Gaussian Mixture model for each location.}
	\label{fig:Responsibilities_M10}
\end{figure}

%%%%%%%%%%%%%%%%%%%%%%%%%%%%%%%%%%%%%%%%%%%%%%%%%%%%%%
\break
\newpage
\subsection{Model Selection}
The scattering of the mixands' responsibilities in the four plots of Fig.~\ref{fig:Responsibilities_M10} indicates that 
the number of clusters $M=10$ might not be the best choice, specially for the locations 
AVS laboratory and ORCCA corridor. In these two models, none of the responsibilities is actually knocked out completely. 
In the light of these results,   model selection is performed next for each location 
through the log likelihood score:% $\mathcal{L}(q)$ 
\begin{equation}
	\mathcal{L}(q) \approx \log P(y_{1:N} \vline M_i)
\end{equation}

\begin{figure}[!h]
	\centering
	\subfigure[Model performance for location: AVS laboratory.]
	{\label{fig:AVS_BIC}
	\includegraphics[scale=0.9]{FiguresC/AVS_BIC}} 
	%
	\subfigure[Model performance for location: ORCCA laboratory.]
	{\label{fig:L_ORCCA} 
	%
	\includegraphics[scale=0.9]{FiguresC/ORCCA_BIC}} 
	\subfigure[Model performance for location: corridor office.]
	{\label{fig:corr_office_BIC}
	\includegraphics[scale=0.9]{FiguresC/corr_office_BIC}} 
	%
	\subfigure[Model performance for location: corridor ORCCA.]
	{\label{fig:corr_ORCCA_BIC}
	\includegraphics[scale=0.9]{FiguresC/corr_ORCCA_BIC}} 
	%
	\caption{Model selection for each location.}
	\label{fig:ModelSelection}
\end{figure}

For every location, the log likelihood is computed considering 
$$M = [2, 3, 4, 5, 6, 7, 8].$$ The model with the highest score is the best fit for the given data. 
A graphical comparison of the different cluster models in each location is provided in Fig.~\ref{fig:ModelSelection}. 
In accordance with these results, %of the plots in Fig.~\ref{fig:ModelSelection}, 
a total number of $M = 6$ clusters is picked for the AVS laboratory model, 
$M = 4$ is chosen for the ORCCA laboratory model, 
$M = 4$ for the corridor office model and 
$M = 4$ for the corridor ORCCA model.

Fig.~\ref{fig:LowerBound} shows the evolution of the lower bound evaluation until convergence for the 
chosen model in each location. 

\begin{figure}[!h]
	\centering
	\subfigure[$L(q)$ for location: AVS laboratory.]
	{\label{fig:L_AVS}
	\includegraphics[scale=0.95]{FiguresC/Lbound_AVS}} 
	%
	\subfigure[$L(q)$ for location: ORCCA laboratory.]
	{\label{fig:L_ORCCA} 
	%
	\includegraphics[scale=0.95]{FiguresC/Lbound_ORCCA}} 
	\subfigure[$L(q)$ for location: corridor office.]
	{\label{fig:L_corr_office}
	\includegraphics[scale=0.95]{FiguresC/Lbound_corridorOffice}} 
	%
	\subfigure[$L(q)$ for location: corridor ORCCA..]
	{\label{fig:L_corr_ORCCA}
	\includegraphics[scale=0.95]{FiguresC/Lbound_corridorORCCA}} 
	%
	\caption{Lower bound convergence for each location.}
	\label{fig:LowerBound}
\end{figure}

From these results it is observed that: 
\begin{itemize}
	\item{AVS laboratory (M=6): }
	Converged at iteration 18.
	\item{ORCCA laboratory(M=4): }
	Converged at iteration 26.
	\item{Corridor office(M=4): }
	Converged at iteration 31.
	\item{Corridor ORCCA(M=4): }
	Converged at iteration 26.
\end{itemize}

Of course now, after model selection, all the mixture components in the chosen model for each location 
will have non-zero responsibility. Figure~\ref{fig:Responsibilities} shows plots of the 
normalized responsibilities for each location model. 

\begin{figure}[!h]
	\centering
	\subfigure[$N_k$ for location: AVS laboratory. M = 6.]
	{\label{fig:Nk_AVS}
	\includegraphics[]{FiguresC/Nk_norm_AVS}} 
	\subfigure[$N_k$ for location: ORCCA laboratory. M = 4.]
	{\label{fig:Nk_ORCCA} 
	\includegraphics[]{FiguresC/Nk_norm_ORCCA}} 
	\subfigure[$N_k$ for location: corridor office. M = 4]
	{\label{fig:Nk_corr_office}
	\includegraphics[]{FiguresC/Nk_norm_corr_office}} 
	\subfigure[$N_k$ for location: corridor ORCCA. M = 4]
	{\label{fig:Nk_corr_ORCCA}
	\includegraphics[]{FiguresC/Nk_norm_corr_ORCCA}} 
	%
	\caption{Normalized responsibilities $N_k$ of each mixand $k$ in the Gaussian Mixture model for each location.}
	\label{fig:Responsibilities}
\end{figure}









